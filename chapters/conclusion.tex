%%%%%%%%%%%%%%%%%%%%%%%%%%%%%%%%%%%%%%%%%%%%%%%%%%%%%%%%%%%%%%%%%%%%%%%%%%%
\chapter{Conclusion}
%%%%%%%%%%%%%%%%%%%%%%%%%%%%%%%%%%%%%%%%%%%%%%%%%%%%%%%%%%%%%%%%%%%%%%%%%%%

The SCS problem is NP-Complete but is approximable.
DNA sequence assembly can be approached as an SCS problem by representing
reads or sequence fragments in an overlap graph.
The greedy algorithm in finding an approximate solution to the SCS problem
is simple and has a constant performance guarantee. Our result shows, 
the greedy algorithm was successful in deriving superstrings with 
an average length less than the conjectured factor-2 performance guarantee.
A genetic algorithm is also viable alternative to the greedy approach,
however, as our results have shown, the complexity in terms of defining and estimating
appropriate parameters and the time it would take for one to extract satisfactory solution, 
may be a great hindrance in the practical use of the technique
such as in DNA sequence assembly.
As a proof of concept, we have shown that indeed the genetic algorithm approach
in solving the SCS problem converges. That is, a better solution
is yield as time or the number of generations is increased. 