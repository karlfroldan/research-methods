%%%%%%%%%%%%%%%%%%%%%%%%%%%%%%%%%%%%%%%%%%%%%%%%%%%%%%%%%%%%%%%%%%%%%%%%%%%
\chapter{Review of Related Systems and Related Literature}
%%%%%%%%%%%%%%%%%%%%%%%%%%%%%%%%%%%%%%%%%%%%%%%%%%%%%%%%%%%%%%%%%%%%%%%%%%%

Several researches have provided in depth study on approximation
algorithms for the SCS problem.

Most \cite{Turner1989, Ma2009, Zaritsky2004}, however, considered a greedy algorithm to solve the SCS problem.
The reason why greed works for shortest common superstring problem was
explored by \cite{Ma2009}. They explained the good performance 
of the greedy algorithm by using the smoothed analysis. 
That is, for any  given instance $I$ of $SCS$, 
the average approximation ratio of the 
greedy algorithm on a small random perturbation of $I$ is $1+o(1)$. 

Apart from the greedy approach, other methods including bio-inspired
algorithms, have also been employed to solve lots of hard problems in 
computer science including the SCS problem. 
In \cite{Zaritsky2004}'s work, for example, four approaches for 
finding solutions to the SCS problem: a 
standard genetic algorithm, a novel cooperative-coevolutionary 
algorithm, a benchmark greedy algorithm, and a parallel 
coevolutionary-greedy approach have been compared. In the paper,
the coevolutionary approach produced the best results.

Zaritsky et al's \cite{Zaritsky2004} work served as the 
foundation of this paper. Hence, the following algorithms,
which have been explored and tested in the latter, have
been extracted and validated 
whether they are indeed generating
good approximate solution to the SCS problem.

\subsection{Greedy-SCS}

The core idea of the greedy algorithm is to repeatedly merge
 pairs of distinct strings with maximum overlap until only one remains.
 It has been conjectured by \cite{MaierStorer1777} that the superstring
 produced by the greedy algorithm is at most two times the optimal.
Below is a pseudocode for the greedy algorithm:

\begin{algorithm}[H]
	\begin{algorithmic}[1]
	\REQUIRE $R$ - set of strings
	\ENSURE superstring of set $R$
	\WHILE{||$R$|| > 1}
		\STATE choose $x_1 \neq x_2 \in R$ such that $overlap(x_1,x_2)$ is maximal 
		\STATE $R \leftarrow (R - \{x_1, x_2\}) \cup \{merge(x_1, x_2)\}$
	\ENDWHILE
	\RETURN remaining string in $R$
	\end{algorithmic}
	\caption{GREEDY(R)}
	\label{alg:seq}
\end{algorithm}

The greedy algorithm is probably the most widely used
heuristic used in DNA-sequencing due to its simplicity and 
constant performance guarantee which is most likely to be factor-2.

\subsection{Genetic Algorithm}

In this paper, we follow the standard model for the genetic algorithm (see Algorithm \ref{alg:gga}).
That is, the algorithm generates an initial population of 
random candidate solutions, then 
the process of selection based on a defined fitness function,
crossover, and mutation are used to evolve the next generation, of which
each individual is then evaluated and
assigned a fitness value. These steps are repeated a
predefined number of times or until the solution is
satisfactory. 


\begin{algorithm}[H]
	\begin{algorithmic}[1]
	\STATE $t \leftarrow 0$ \{generation number\} 
	\STATE initialize Population$_{t}$ 
	\STATE evaluate(Population$_{t}$)
	\WHILE{termination condition not met}
		\STATE select individuals from Population$_{t}$ 
		\STATE recombine individuals
		\STATE mutate individuals	
		\STATE Population$_{t+1} \leftarrow $ newly created individuals
		\STATE $t \leftarrow t + 1$
		\STATE evaluate(Population$_{t}$) 			
	\ENDWHILE
	\RETURN solution derived from the best individual in Population$_{t}$
	\end{algorithmic}
	\caption{GENERIC GA()}
	\label{alg:gga}
\end{algorithm}
