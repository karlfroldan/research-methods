%%%%%%%%%%%%%%%%%%%%%%%%%%%%%%%%%%%%%%%%%%%%%%%%%%%%%%%%%%%%%%%%%%%%%%%%%%%
\chapter{Technical Background}
%%%%%%%%%%%%%%%%%%%%%%%%%%%%%%%%%%%%%%%%%%%%%%%%%%%%%%%%%%%%%%%%%%%%%%%%%%%

A superstring is simply a string over
some alphabet for which given a set of string from the same
alphabet, the latter's members are all substrings of the former.
To understand the SCS problem, it is best to assume some important concepts
related to the problem itself. First, let
us assume some set $R = \{x_1, \ldots , x_n\}$ as
 a set of strings (or blocks) over some alphabet $\Sigma$. 
Formally, we define a $superstring$ of $R$ as a 
string $w$ containing each $x_i \in R$, as a substring.

Following are the elementary operations associated 
with the construction of a superstring.
The $overlap(u, v)$ of two strings $u$ and $v$ 
is the maximum overlap between $u$ and $v$. That is to say,
the longest suffix of $u$ (in terms of characters) that is a prefix of $v$.
The $prefix(u, v)$ of $u$ and $v$ is the prefix of $u$
obtained by removing its overlap with $v$.
Lastly, $merge(u, v)$ is the
 concatenation of $u$ and $v$ with the overlap appearing
only once.

As an example, say we have, $\Sigma = \{a, b, c\}$ and $R = \{cbcaca, cacac\}$.
The string $cacacbcaca$ is a superstring of $R$ while the string
$cbcacac$ is the shortest common superstring. The
following relations also hold: 

\begin{tabular}{p{0.5cm}l}
&$overlap(cbcaca, cacac) = caca$, \\
&$overlap(cacac, cbcaca) = c$ ,\\
&$prefix(cbcaca, cacac) = cb$,\\
&$merge(cbcaca, cacac) = cbcacac$.
\end{tabular}

A $superstring$ $w$ is also defined as
 the string $prefix(x_1, x_2) \cdot prefix(x_2, x_3) \cdots prefix(x_n, x_1) \cdot overlap(x_n, x_1)$.
That is, a $superstring$ is simply
 a concatenation of all the strings ``minus'' the overlapping duplicates.
 Apparently, each superstring of a set of strings defines a permutation of the set’s elements.
Conversely, every permutation of the set’s elements corresponds to a single superstring
			 
Our interest however lies on defining the SCS problem.
Essentially, the SCS problem is: Given a set of strings $R$
and a positive integer $K$, does $R$ have a superstring of length $K$?

Figure \ref{fig:scsdp} shows the SCS Decision Problem following the template
provided by Garey and Johnson \cite{GareyJohnson1979}.

\begin{figure}[ht!]
\centering
\fbox{
\begin{tabular}{p{7.5cm}}

INSTANCE: \\
Finite alphabet $\Sigma$, finite set $R$ of strings from $\Sigma^*$ and a positive integer $K$. \\
QUESTION: \\
Is there a string $w \in \Sigma^*$ with $\vert w\vert \leq K$ such that each string $x\in R$ is a substring of w, 
i.e. $w=w_0 x w_1$ where  $w_0,w_1\in\Sigma^*$? \\
\end{tabular}
}
\caption{SCS Decision Problem }
\label{fig:scsdp}
\end{figure}


In the Compendium of NP Optimization problems published by \cite{AusielloEtAl2000} and in the list of 
NP-Complete Problems published by \cite{GareyJohnson1979} the SCS problem appears under under Storage and Retrieval (SR).