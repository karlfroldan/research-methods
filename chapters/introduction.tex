%%%%%%%%%%%%%%%%%%%%%%%%%%%%%%%%%%%%%%%%%%%%%%%%%%%%%%%%%%%%%%%%%%%%%%%%%%%
\chapter{Introduction}
%%%%%%%%%%%%%%%%%%%%%%%%%%%%%%%%%%%%%%%%%%%%%%%%%%%%%%%%%%%%%%%%%%%%%%%%%%%

Since 2010, functional programming has been getting more 
popular (Purescript, Typescript, ReactJS, Elm), people are 
trying to find more ways of leveraging the relatively easier-reasonability 
of functional programming without having to sacrifice 
program performance. Directly proportional to the trend is 
the developing advancements in time and space complexity 
of different programming languages and those events are 
inevitable due to the fact that we live in a technological 
age. This paper’s goal is to know how can a lazy-first 
purely functional programming language have comparable 
performance with respect to an eager memory-safe language 
in terms of multi-threaded programming? In turn helps in 
the said advancements in different programming languages.

% Functional programming has suddenly risen to popularity 
% with examples that include Scala, Clojure, ReactJS, and 
% other languages adopting lambda expressions. Reasoning 
% about program correctness in a pure function can be done 
% in a dependently-typed, proof assistant such as Coq or 
% Agda\cite{Breitner2018}\cite{SpectorZabusky2018}\cite{ElBakouny2017}. 
% Likewise, pure functions can be easily proved by using 
% induction. Composing two proven functions into a single 
% function should also give the correct result\cite{AbelBenkeBove2005}.

% This research aims to utilize the existing A* pathfinding 
% algorithm\cite{ZaghloulAlJami2017}\cite{WeinstockHolladay}
% and find a way to develop a reasonably-efficient 
% purely functional implementation of the algorithm using parallel 
% data structures such as STMs or MVars\cite{Marlow2013}.  
% The main objective of the research is 
% to find an efficient concurrent implementation of a maze solver 
% in a purely functional programming environment with comparable 
% performance and space complexity of a performant imperative 
% programming language.

% The Shortest Common Superstring (SCS) problem, known to be NP-Complete,
% seeks the shortest string that contains all strings from a given set.
% In this paper, we provide the summary of the problem and some of its characteristics.

% The SCS problem has been extensively studied for its
% applications in string compression and DNA sequence assembly \cite{Ma2009}.

% The superstring problem has applications to data storage,
%  specifically, data (string) compression \cite{Gallant1980}. 
% In many programming languages, a character string may be 
% represented by a pointer to that string. 
% The problem for the compiler is to arrange strings 
% so that they may be ``overlapped'' as much as possible.

% DNA sequence assembly is another  problem to which an SCS algorithm is known to apply.
% The $sequencing$ problem in molecular biology is to ``read'' a string of DNA,
% which can be viewed as a string over the alphabet \{A,C,G,T\}. Sequencing produces such a large number of fragments that
% almost all genome positions are covered by many fragments. This short fragments
% thus have large overlaps between other pieces. Hence, they can be given as an input to SCS algorithm.
% Figure \ref{fig:dna-overlap} shows an overlap graph consisting DNA reads (or fragments) as nodes. 



% \begin{figure*}
% \centering
% \fbox{
% \scalebox{0.65}{
% \includegraphics{fig-overlaps-dns-example}
% }}
% \caption{Sample overlap graph with each adjacent nodes 
% having at least $k = 3$ overlaps. The original string is \texttt{GCATTATATATTGCGCGTACGGCGCCGCTACA}.}
% \label{fig:dna-overlap}
% \end{figure*}	

% In \cite{Ma2009}'s paper, SCS was used to analyze DNA sequence assembly using
% a greedy algorithm. 

\section{Project Context}

Reasoning about program correctness in a pure function can be done 
in a dependently-typed, proof assistant such as Coq or 
Agda\cite{Breitner2018}\cite{SpectorZabusky2018}\cite{ElBakouny2017}. 
Likewise, pure functions can be easily proved by using 
induction. Composing two proven functions into a single 
function should also give the correct result\cite{AbelBenkeBove2005}.

As more people use functional programming languages, 
the need for pure function algorithms has greatly increased. 
Most references such as \emph{Introduction to Algorithms},
\emph{The Algorithm Design Manual}, and \emph{The Art of Computer Programming} 
present algorithms in a mostly imperative or structural manner.\cite{CLRS}\cite{Skiena}\cite{Knuth1997} 
Thus, most programmers are more familiar with imperative 
approaches to programming.


\section{Purpose and Description}

This research aims to utilize the existing A* pathfinding algorithm
\cite{ZaghloulAlJami2017}\cite{WeinstockHolladay}
and find a way to develop a reasonably-efficient purely-functional 
implementation of the algorithm using parallel data structures such 
as STMs or MVars\cite{Marlow2013}.  

The A* Pathfinding algorithm is used heavily in video games, telephone traffic, 
and other graph traversal problems\cite{HartNilssonRaphael1968}. This research 
aims to aid in the development of video games using the functional 
programming paradigm in the future as video game development is dominated 
by imperative languages.

\section{Objectives}
The main objective of the research is to find an efficient concurrent 
implementation of a maze solver in a purely-functional programming 
environment with comparable performance and space complexity with 
respect to an efficient imperative programming language. The 
researchers also aim to compare performances between Haskell and Rust 
by creating a program such as a maze-solver that can be benchmarked using 
both programming languages, then gather necessary information and come up with a 
detailed analysis regarding the performances of both languages where the number 
of CPU cores and threads as independent variables.


% \vfill\eject
\section{Scope and Limitations}
The researchers will utilize Haskell for concrete implementation 
of the parallel A* algorithm in a functional programming environment.
Likewise, for performance comparison, the researchers 
will use the Rust programming language due to some of its features having
similarities with Haskell such as correct concurrent programs\cite{Saligrama2019}
and guarantees a relatively safe program\cite{Jung2018}.

Other purely-functional languages or lambda notation for generality
will not be used. Other concurrent data structures besides \verb|MVar| and Software
Transactional Memory will not be utilized. Likewise, only mazes with
a reachable end-goal will be tested since if the A* algorithm runs 
on unbounded mazes, that is - having no reachable end goal, the 
algorithm will not halt.\cite{HartNilssonRaphael1968}

The concrete implementation and analysis is planned to be tested on only one CPU, namely
Intel Core i7-9750H CPU at 2.60Ghz with 6 cores and 12 logical threads.