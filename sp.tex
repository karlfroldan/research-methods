%%%%%%%%%%%%%%%%%%%%%%%%%%%%%%%%%%%%%%%%%%%%%%%%%%%%%%%%%%%%%%%%%%%%%%%%%%%
% filename    : sp.tex
% author      : 
%%%%%%%%%%%%%%%%%%%%%%%%%%%%%%%%%%%%%%%%%%%%%%%%%%%%%%%%%%%%%%%%%%%%%%%%%%%

% BUGFIX: Save LaTeX kernel version of \@xfloat
\makeatletter
\let\my@xfloat\@xfloat
\makeatother

% use ateneo de naga etd class (actually modified virginia tech's etd class)
\documentclass[oneside]{etd}

% BUGFIX: Create a modified copy of \@xfloat using the kernel definition
\makeatletter
\def\@xfloat#1[#2]{
	\my@xfloat#1[#2]%
	\def\baselinestretch{1}%
	\@normalsize \normalsize
}
\makeatother

% use these packages

\usepackage{graphicx}
\usepackage{amssymb}
\usepackage{gensymb}
\usepackage{latexsym}
\usepackage{amsmath}
%\usepackage[lineno5,noindent]{lgrind}
%\usepackage{rotating} 
%\usepackage{makeidx}
%\usepackage{stmaryrd}
\usepackage{float}
\usepackage{subfigure}
%\usepackage{cite}
%\usepackage{moreverb}
%\usepackage{pictexwd,dcpic}
\usepackage{algorithm,algorithmic}
\renewcommand{\algorithmicrequire}{\textbf{Input:}}
\renewcommand{\algorithmicensure}{\textbf{Output:}}
\usepackage{tikz}
\usetikzlibrary{positioning,calc}

\usepackage{listings}
\definecolor{cppred}{rgb}{0.6,0,0} % for strings
\definecolor{cppgreen}{rgb}{0.25,0.5,0.35} % comments
\definecolor{cpppurple}{rgb}{0.5,0,0.35} % keywords
\definecolor{cppdocblue}{rgb}{0.25,0.35,0.75} % doc
 
\lstset{language=C++,
basicstyle=\linespread{0.8}\ttfamily\small,
keywordstyle=\color{cpppurple}\bfseries,
stringstyle=\color{cppred},
commentstyle=\color{cppgreen},
morecomment=[s][\color{cppdocblue}]{/**}{*/},
numbers=left,
numberstyle=\tiny\color{black},
% stepnumber=2,
numbersep=10pt,
tabsize=4,
showspaces=false,
showstringspaces=false}

\graphicspath{{figures/}{./}}

%\renewcommand{\floatpagefraction}{0.8}

\makeindex

%%%%%%%%%%%%%%%%%%%%%%%%%%%%%%%%%%%%%%%%%%%%%%%%%%%%%%%%%%%%%%%%%%%%%%%%%%%

\title{Senior Project Documentation Title}
\author{My Name}
\department{Computer Science}
\BSdegree
\field{Information Technology}
\degreemonth{Month Day}  % date of final defense
\degreeyear{2020}

\threestudents
{First Student}
{Second Student}
{Third Student}

\threestudentsheader
{Surname1, Surname2, and Surname3}

\threedegrees
{Information Technology}
{Computer Science}
{Information Systems}


\advisor
{Adviser's Name}
\threemembers
{First Panel Member}
{Second Panel Member}
{Third Panel Member}
\deanandchair
{Joshua C. Martinez, MIT}
{Marianne P. Ang, MS}

\keywords{keywordOne, keywordTwo, keywordThree}


\begin{document}

\maketitle
\makerecomm
\makeacceptance
\makedeclaration

%%%%%%%%%%%%%%%%%%%%%%%%%%%%%%%%%%%%%%%%%%%%%%%%%%%%%%%%%%%%%%%%%%%%%%%%%%%
%                    ABSTRACT/EXECUTIVE SUMMARY                           %
%%%%%%%%%%%%%%%%%%%%%%%%%%%%%%%%%%%%%%%%%%%%%%%%%%%%%%%%%%%%%%%%%%%%%%%%%%%

\begin{execsummary}
	The Shortest Common Superstring (SCS) problem, known to be NP-Complete,
	seeks the shortest string that contains all strings from a given set.
	This paper provides a detailed summary of SCS problem including the problem's 
	complexity and approximability, and
	shows that DNA sequence assembly be can approached using SCS.
	Then this paper discusses the two most popular approximation algorithms for the 
	SCS problem namely greedy and
	genetic algorithm. Finally, this study presents results supporting the simplicity and performance
	guarantee of the greedy algorithm as well as the convergence of the genetic 
	algorithm to satisfactory approximate solutions as indivuals representing
	problem instances evolve.	
\end{execsummary}

	
\begin{dedication}

I dedicate this research work to all of humanity.

\end{dedication}


\begin{acknowledgements}

I thank everyone who helped me finish this thesis.

\end{acknowledgements}


\begingroup
\renewcommand*{\addvspace}[1]{}
\tableofcontents
\listoffigures
\listoftables
\endgroup

\beginbody
\pagenumbering{arabic}

%%%%%%%%%%%%%%%%%%%%%%%%%%%%%%%%%%%%%%%%%%%%%%%%%%%%%%%%%%%%%%%%%%%%%%%%%%%
\chapter{Introduction}
%%%%%%%%%%%%%%%%%%%%%%%%%%%%%%%%%%%%%%%%%%%%%%%%%%%%%%%%%%%%%%%%%%%%%%%%%%%
The A* pathfinding algorithm is a best-first pathfinding algotithm for graphs commonly used 
for graph traversal applications such as artificial intelligence, video games, flight paths, and more.
However, while most games are written in an imperative and object-oriented language such as C\#, C++, and 
JavaScript, it is possible to write video games in a functional language using a reactive 
functional programming approach.\cite{Cheong2006} Likewise, the need for other correct critical software 
led organizations such as NASA to use Haskell\cite{HaskellSite}, a purely-functional programming language, to 
be used in systems where high-level assurance and provable programs are a must.\cite{NasaCopilot2020}

This paper assumes concrete differences between \emph{parallel} and \emph{concurrent} where the former 
is defined to be a hardware feature of having multiple processors or cores to compute a problem whereas the 
latter is defined to be a software-based approach to decrease the impact of computation bottlenecks by 
switching between different computations when a computation takes too long.\cite{SilberschatzGalvin2012} One of the major challenges 
of parallel programming is controlling the order of execution to prevent \emph{race conditions}, 
which can often lead to bugs and are hard to maintain. However, since pure functional languages, such as Haskell,
have no mutability and computations lead to the same result regardless of the order, they are a perfect candidate 
for writing parallel programs.\cite{Hammond2011}
This research aims to find a parallel implementation of the existing A* pathfinding algorithms using a 
purely-functional setting with attention to program performance. In turn, this helps in the advancement 
of different functional programming approaches for parallel graph computations which could eventually lead
to critical systems to use a more provable programming language.


% The Shortest Common Superstring (SCS) problem, known to be NP-Complete,
% seeks the shortest string that contains all strings from a given set.
% In this paper, we provide the summary of the problem and some of its characteristics.

% The SCS problem has been extensively studied for its
% applications in string compression and DNA sequence assembly \cite{Ma2009}.

% The superstring problem has applications to data storage,
%  specifically, data (string) compression \cite{Gallant1980}. 
% In many programming languages, a character string may be 
% represented by a pointer to that string. 
% The problem for the compiler is to arrange strings 
% so that they may be ``overlapped'' as much as possible.

% DNA sequence assembly is another  problem to which an SCS algorithm is known to apply.
% The $sequencing$ problem in molecular biology is to ``read'' a string of DNA,
% which can be viewed as a string over the alphabet \{A,C,G,T\}. Sequencing produces such a large number of fragments that
% almost all genome positions are covered by many fragments. This short fragments
% thus have large overlaps between other pieces. Hence, they can be given as an input to SCS algorithm.
% Figure \ref{fig:dna-overlap} shows an overlap graph consisting DNA reads (or fragments) as nodes. 



% \begin{figure*}
% \centering
% \fbox{
% \scalebox{0.65}{
% \includegraphics{fig-overlaps-dns-example}
% }}
% \caption{Sample overlap graph with each adjacent nodes 
% having at least $k = 3$ overlaps. The original string is \texttt{GCATTATATATTGCGCGTACGGCGCCGCTACA}.}
% \label{fig:dna-overlap}
% \end{figure*}	

% In \cite{Ma2009}'s paper, SCS was used to analyze DNA sequence assembly using
% a greedy algorithm. 

\section{Project Context}
The Therac-25 medical radiation machine caused at overdose accidents in which it harmed at least six patients using the machine.
One of the main causes of the problem was due to poor software engineering practices and that testing the software was
not enough.\cite{Therac1}
Most critical systems such as flight software, navigation, and military software is written in languages that may not guarantee 
correctness of programs such as C where adding $1$ to an \verb|int8| whose value is $65535$ may cause the program to show the 
incorrect sum of $-65536$. These kinds of problems often lead to software crashes and cause unexpected failures.
Also, the video game industry is dominated by imperative programming languages. However, due to the rise of 
languages such as TypeScript and PureScript, and frameworks such as ReactJS which favors a reactive functional programming approach,
it is not fanciful to say that games might start being written in the functional style as well. 

However, the functional style of programming has a different set of problems compared to its imperative counterpart such as 
obscurity for some programmers (though, this is subjective) and increased space complexity for the same algorithm. Multiple 
algorithms for the imperative approaches may not translate well and may even increase the time and space complexity for the functional 
approach due to its nature of copying the data structure instead of iterating over.


The advantages of functional programming lies with its \emph{referential transparency} which means that 
a function definition or a variable will never change its definition throughout the runtime of the program.\cite{Kesseler1996,Hammond2011}
Hence, mathematically proving functional programs might be easier and can be aided by proof assistants such as  
Coq or Agda.\cite{Breitner2018,SpectorZabusky2018,ElBakouny2017} Likewise, splitting functions into smaller functions and 
reasoning about those smaller components much like lemmas would mean that functions would be modular and composed of 
proven subfunctions.\cite{AbelBenkeBove2005,Hughes1989} Hence, functional programming languages are 
excellent candidates for parallel programming since the languages do not have mutable states and are therefore, instances of
shared variables are abstracted away from the programmer. Similarly, the order of execution of pure functions does not matter 
as the program will still yield the same results.\cite{Kesseler1996} 
A parallel and purely-functional approach to the A* algorithm would eventually lead to more applications such as
shortest distance in a map, flight paths, web server searching, and more to use a more provable and type-safe language which 
could lead to less system failures and high availability.

\section{Purpose and Description}

This research aims to utilize the existing parallel A* pathfinding algorithm
\cite{ZaghloulAlJami2017,WeinstockHolladay}
and find a way to develop a reasonably-efficient purely-functional 
implementation of the algorithm using parallel data structures such 
as STMs or MVars\cite{Marlow2013}.  

The A* Pathfinding algorithm is used heavily in video games, telephone traffic, 
and other graph traversal problems\cite{HartNilssonRaphael1968}. This research 
aims to aid in the development of video games and in developing safer critical systems 
with stricter error-checking and type safety.\cite{NasaCopilot2020} 

\section{Objectives}
The main objective of the research is to develop functional implementations of two different
parallel A* implementations. The research will be done using Haskell and Rust as a comparative metric for
imperative languages. Likewise, concrete comparisons between the number of cores and logical threads will be 
used to measure the most efficient runtime and space complexity of both algorithms.

The researchers aim to complete the following tasks:
\begin{itemize}
    \item Model multiple combinatorial problems such as the $n$-queens problem and Sudoku. These problems 
        may or may not have different problem sizes.
    \item A solver will be written both in Haskell, a laze purely-functional programming language, and Rust, 
        a relatively modern systems programming language that shares multiple features with Haskell.
    \item Two algorithms will be written in both languages and their performance will be recorded. The program
        Threadscope will be utilized for recording the performance of the Haskell solvers, such as thread and core 
        activities while the program is being run.
\end{itemize}

% The main objective of the research is to find an efficient parallel purely-functional implementation 
% of the A* pathfinding algorithm. The research will be done mostly in Haskell with some exceptions.
% Likewise, concrete comparisons between the number of cores and logical threads will be used to measure 
% the most efficient performance runtime and space complexity of the algorithm.

% The researchers aim to complete the following specific tasks:
% \begin{itemize}
%     \item Write a \emph{generator} that will generate an arbitrary-sized maze. The maze should be relatively 
%         hard to solve without the aid of computers in a short amount of time.\cite{Buck2015}
%     \item And a \emph{solver} program that will be written in Haskell, a lazy purely-functional programming language,
%         for translating the output of the generator to a graph.\cite{HaskellSite}
%     \item The solver program should have a web-based user interface where the researchers can view the maze and how 
%         the solver program was able to solve it correctly.
%     \item Performance of the solution shall be measured by using ThreadScope to monitor the thread and core activities 
%         while the program is being run.\cite{ThreadScope}
% \end{itemize}
% \vfill\eject
\section{Scope and Limitations}

The research will only cover Haskell, though it may generalize to other functional languages that support a parallel 
and concurrent approach. Translation to other functional programming languages is not a priority and thus, the use 
of abstract machines or lambda calculus notation will not be used. The researches deem that using a pure functional language 
such as Haskell will enable it to generalize well even on impure languages such as LISP.
Also, only problems where solutions exist will be tested on. 

The concrete implementation and analysis is planned to be tested only on four CPUs such as Intel Core i7-9750H and AMD Ryzen 5 3500x.
Other CPU architectures are not planned to be tested on.
%%%%%%%%%%%%%%%%%%%%%%%%%%%%%%%%%%%%%%%%%%%%%%%%%%%%%%%%%%%%%%%%%%%%%%%%%%%
\chapter{Review of Related Systems and Related Literature}
%%%%%%%%%%%%%%%%%%%%%%%%%%%%%%%%%%%%%%%%%%%%%%%%%%%%%%%%%%%%%%%%%%%%%%%%%%%

Due to the trend of parallelization in the modern computing era, \cite{Rios2011,WeinstockHolladay,ZaghloulAlJami2017,Sanz2016}
several researchers developed more efficient and faster implementations 
of the A* algorithm by parallizing the algorithm in various ways.

\section{The A* Algorithm}
Here, we shall describe the original A* algorithm by Hart, Nilsson, and Raphael.\cite{HartNilssonRaphael1968}
A pure-function pseudocode\ref{seqAStar} will be implemented instead of the usual imperative 
pseudocode. But the pseudocode still follows the specifications of the original paper.

\begin{algorithm}
    \caption{Sequential A* Algorithm}
    \label{seqAStar}
    \begin{algorithmic}
        \REQUIRE $G$ - a weighted labeled graph.
        \REQUIRE $(\psi, \xi)$ - a tuple of two sets: the open set and the closed set
        \REQUIRE $h$ - the heuristic function. $h(n)\neq 0$ or this equates to Dijkstra's Algorithm.
        \REQUIRE $(\sigma, \gamma)$ - a tuple of two vertices in $G$: the start vertex and end vertex.
        \ENSURE  $p$ - The shortest path from $\psi$ to $\xi$. 
        
        \STATE let $S$ be a homogenous set of vertices in a graph.
        \STATE astar :: Graph $\rightarrow$ ($S$, $S$) $\rightarrow$ ((Int, Int) 
            $\rightarrow$ (Int, Int) $\rightarrow$ Int) $\rightarrow$ (Vertex, Vertex) $\rightarrow$ Path 
        \STATE astar g s h se p = \textbf{WRITE THE ALGORITHM LATER}.
    \end{algorithmic}
\end{algorithm}

% \begin{algorithm}[H]
% 	\begin{algorithmic}[1]
% 	\REQUIRE $R$ - set of strings
% 	\ENSURE superstring of set $R$
% 	\WHILE{||$R$|| > 1}
% 		\STATE choose $x_1 \neq x_2 \in R$ such that $overlap(x_1,x_2)$ is maximal 
% 		\STATE $R \leftarrow (R - \{x_1, x_2\}) \cup \{merge(x_1, x_2)\}$
% 	\ENDWHILE
% 	\RETURN remaining string in $R$
% 	\end{algorithmic}
% 	\caption{GREEDY(R)}
% 	\label{alg:seq}
% \end{algorithm}

\section{Parallel A* Algorithm}
Here, we shall discuss the naive implementation of Zaghloul\cite{ZaghloulAlJami2017}
where the researchers parallelized A* using MISD and compared the running times and speed up 
of the algorithm in solving some problems based ont he number of cores of the computer it was run on.

Also, we shall visit the implementations by Rios and Chaomowicz\cite{Rios2011}
which used a more enhanced implementation of the single-core bidirectional pathfinding algorithm by 
Kaindl and Kainz\cite{KainlKainz1997} by parallelizing the two directions on two CPU cores.

Likewise, we shall also visit HDA* by Kishimoto, Fukunaga, and Botea\cite{Kishimoto2009} on how it 
assigned ownership of vertices between different CPU cores by hashing and the performance comparison of
Holladay and Weinstock\cite{WeinstockHolladay}.

Lastly, we compare how the HDA* may be a problem for Haskell since 
Haskell's parallelism is abstracted. Instead, the paper will define 
another method of parallelizing the A* algorithm on a multicore machine 
using Haskell.

\section{Inductive Graphs}
This section will discuss the \verb|functional graph library| as described by Erwig.\cite{Erwig2001}
Inductive graphs can be constructed monadically which will be of advantage when writing 
parallelizing an algorithm since parallel code can only be executed monadically.

We use both inductive graphs for the sequential implementation (not a monad) and for the parallel
implementation (monad).

\section{Real-time Performance of Functional Programs}
In this section, we compare the usefulness of functional programs in real-time situations, that is, in 
applications were time is critical such as GPS systems and flight guidance software.\cite{Frame2014}
Since we're primarily concerned with developing a fast-enough parallel program for a correct software that 
would be needed for critical systems, we have to compare the performance of our program with recent imperative 
implementations. This should be stated here.

Likewise, we make use of Tim Harris, Simon Marlow, and Simon Peyton Jones' research on the parallel 
performance of Haskell and overhead costs when using parallel data structures.\cite{Harris2005}


% Several researches have provided in depth study on approximation
% algorithms for the SCS problem.

% Most \cite{Turner1989, Ma2009, Zaritsky2004}, however, considered a greedy algorithm to solve the SCS problem.
% The reason why greed works for shortest common superstring problem was
% explored by \cite{Ma2009}. They explained the good performance 
% of the greedy algorithm by using the smoothed analysis. 
% That is, for any  given instance $I$ of $SCS$, 
% the average approximation ratio of the 
% greedy algorithm on a small random perturbation of $I$ is $1+o(1)$. 

% Apart from the greedy approach, other methods including bio-inspired
% algorithms, have also been employed to solve lots of hard problems in 
% computer science including the SCS problem. 
% In \cite{Zaritsky2004}'s work, for example, four approaches for 
% finding solutions to the SCS problem: a 
% standard genetic algorithm, a novel cooperative-coevolutionary 
% algorithm, a benchmark greedy algorithm, and a parallel 
% coevolutionary-greedy approach have been compared. In the paper,
% the coevolutionary approach produced the best results.

% Zaritsky et al's \cite{Zaritsky2004} work served as the 
% foundation of this paper. Hence, the following algorithms,
% which have been explored and tested in the latter, have
% been extracted and validated 
% whether they are indeed generating
% good approximate solution to the SCS problem.

% \subsection{Greedy-SCS}

% The core idea of the greedy algorithm is to repeatedly merge
%  pairs of distinct strings with maximum overlap until only one remains.
%  It has been conjectured by \cite{MaierStorer1777} that the superstring
%  produced by the greedy algorithm is at most two times the optimal.
% Below is a pseudocode for the greedy algorithm:

% \begin{algorithm}[H]
% 	\begin{algorithmic}[1]
% 	\REQUIRE $R$ - set of strings
% 	\ENSURE superstring of set $R$
% 	\WHILE{||$R$|| > 1}
% 		\STATE choose $x_1 \neq x_2 \in R$ such that $overlap(x_1,x_2)$ is maximal 
% 		\STATE $R \leftarrow (R - \{x_1, x_2\}) \cup \{merge(x_1, x_2)\}$
% 	\ENDWHILE
% 	\RETURN remaining string in $R$
% 	\end{algorithmic}
% 	\caption{GREEDY(R)}
% 	\label{alg:seq}
% \end{algorithm}

% The greedy algorithm is probably the most widely used
% heuristic used in DNA-sequencing due to its simplicity and 
% constant performance guarantee which is most likely to be factor-2.

% \subsection{Genetic Algorithm}

% In this paper, we follow the standard model for the genetic algorithm (see Algorithm \ref{alg:gga}).
% That is, the algorithm generates an initial population of 
% random candidate solutions, then 
% the process of selection based on a defined fitness function,
% crossover, and mutation are used to evolve the next generation, of which
% each individual is then evaluated and
% assigned a fitness value. These steps are repeated a
% predefined number of times or until the solution is
% satisfactory. 


% \begin{algorithm}[H]
% 	\begin{algorithmic}[1]
% 	\STATE $t \leftarrow 0$ \{generation number\} 
% 	\STATE initialize Population$_{t}$ 
% 	\STATE evaluate(Population$_{t}$)
% 	\WHILE{termination condition not met}
% 		\STATE select individuals from Population$_{t}$ 
% 		\STATE recombine individuals
% 		\STATE mutate individuals	
% 		\STATE Population$_{t+1} \leftarrow $ newly created individuals
% 		\STATE $t \leftarrow t + 1$
% 		\STATE evaluate(Population$_{t}$) 			
% 	\ENDWHILE
% 	\RETURN solution derived from the best individual in Population$_{t}$
% 	\end{algorithmic}
% 	\caption{GENERIC GA()}
% 	\label{alg:gga}
% \end{algorithm}

%%%%%%%%%%%%%%%%%%%%%%%%%%%%%%%%%%%%%%%%%%%%%%%%%%%%%%%%%%%%%%%%%%%%%%%%%%%
\chapter{Technical Background}
%%%%%%%%%%%%%%%%%%%%%%%%%%%%%%%%%%%%%%%%%%%%%%%%%%%%%%%%%%%%%%%%%%%%%%%%%%%

\section{Graph Theoretic Preliminaries}
A weighted, directed graph $G=(V,E)$ with $V$ being the set of vertices of $G$ and $E$ being the set 
of edges of $G$ is said to be a graph with an associated weight function $w:E\to\mathbb{R}$ mapping the edges of $G$
to real-valued weights. The $\walk{u}{v}$, where $u,v\in V$ is a sequence of vertices $(u, u_1, u_2,\dots,v)$ where 
consecutive vertices are adjacent to each other in $G$. A $\graphpath{u}{v}$ is a $\walk{u}{v}$ with no repeating vertices,
implicitly, no repeating edges.

In this paper, an edge is denoted by the two vertices incident of the edge. Example, for two vertices $u$, $v$ in a graph, 
an edge between $u$ and $v$ would be denoted by $uv$. Hence, a path $(v_0, v_1, \dots, v_n)$ has edges
$\{v_0v_1, v_1v_2,\dots, v_{n-1}v_n\}$. 

\section{Minimum Cost Paths}
The minimum cost paths is a discrete optimization problem wherein the goal is to determine 
the shortest path from a start vertex in a graph to an end vertex in a graph. This paper only 
deals with single-source shortest path. The shortest path problem can be defined formally as follows.

The function $W$ of a path $W(p)$ where $p$ is a path is denoted by the sum of the edges 
in the path.\cite{CLRS} That is, if some path $p=(v_0, v_1, \dots, v_n)$, the function $W$ is defined as 
\[
W(p) = \sum^k_{i=1} w(v_{i-1}v_i). 
\]

Hence, the shortest path weight can be defined by the function $\zeta: V\to V$ (a mapping from a vertex in $G$ to another vertex in $G$)
as follows 
\[
\zeta(u, v)=\begin{cases}
    \text{min}(S_{u\to v}) & \text{if there is a path from $u$ to $v$} \\ 
    \infty  & \text{otherwise}
\end{cases}    
\]
where $S_{u\to v}$ is the set of weights of all paths from $u$ to $v$, that is, $S_{u\to v}=\{W(p) | p=(u,v_1, v_2,\dots, v)\}$.
Hence, the \textbf{shortest path} is defined as the path with the shortest weight, that is $W(p)=\zeta(u, v)$.\cite{CLRS}

\section{A* Algorithm}
While the shortest path in a graph problem can be solved by dynamic programming, other algorithms exist such as branch and bounding algorithms 
like the A* search.\cite{HartNilssonRaphael1968} The A* search algorithm employs the use of heuristic functions to estimate the cost
or distance from the current vertex to the final vertex, while keeping in mind the current best. Hence, the A* search can be thought of 
as an intelligent algorithm in the classical sense. A heuristic function is \textbf{admissible} if it is proven that it will never 
overestimate the cost of reaching the end vertex. That is, the heuristic function may underestimate or equal the actual cost. 
Assuming the heuristic function is admissible, then the A* algorithm is considered to be admissible. 

Likewise, the heuristic function of the A* algorithm is \textbf{consistent} or monotone if the estimate of the cost from the current 
vertex is always less than or equal to the cost of its neighbors in the open set (vertices that haven't been analyzed) to the end vertex
plus the cost of reaching the neighbor from the current vertex.

\begin{definition}
    If a vertex $v$ is in the closed set, then it is \textbf{expanded}. Likewise, \textbf{expanding} $v$ means adding $v$ to the closed set.
\end{definition}

\begin{algorithm}[H]
    \begin{algorithmic}[1]
        \REQUIRE $G$ - the graph
        \REQUIRE $u$ - start vertex.
        \REQUIRE $v$ - end vertex 
        \STATE Let $\Omega$ be the open set.
        \STATE Let $\Gamma$ be the closed set.
        \STATE Let $f$ be the evaluating function
        \STATE Add $u$ to $\Omega$.
        \WHILE {$\Omega\neq\varnothing$}
            \STATE Get the vertex $a\in\Omega$ with the minimum $f(a)$.
            \IF{$a=v$}
                \STATE Add $a$ to $\Gamma$ and terminate A* search.
            \ELSE 
                \STATE Add $a$ to $\Gamma$
                \STATE Let $N$ be the set of adjacent vertices of $a$ not in the closed set.
                \STATE For all $n\in N$, add $n$ to $\Omega$.
                \STATE For neighbors $m$ of $a$ in the closed set
                \IF {$f(m)$ is lower than the cost when $m$ is expanded}
                    \STATE Unexpand $m$, that is, add $m$ to the open set 
                \ENDIF
            \ENDIF
        \ENDWHILE
    \end{algorithmic}
    \caption{A* Search Algorithm}
\end{algorithm}

% \begin{algorithm}[H]
% 	\begin{algorithmic}[1]
% 	\REQUIRE $R$ - set of strings
% 	\ENSURE superstring of set $R$
% 	\WHILE{||$R$|| > 1}
% 		\STATE choose $x_1 \neq x_2 \in R$ such that $overlap(x_1,x_2)$ is maximal 
% 		\STATE $R \leftarrow (R - \{x_1, x_2\}) \cup \{merge(x_1, x_2)\}$
% 	\ENDWHILE
% 	\RETURN remaining string in $R$
% 	\end{algorithmic}
% 	\caption{GREEDY(R)}
% 	\label{alg:seq}
% \end{algorithm}

\section{Parallel A* Algorithm}
Two parallel methods described will be used in this paper. Specifically, HDA* and PNBA*.\cite{Kishimoto2009,Rios2011}

\subsection{PNBA*}
The PNBA* is a parallelization of the bidirection A* algorithm. One of the main advantages of computing the shortest path 
bidirectionally is that it reduces the search space and search effort.\cite{KainlKainz1997,Pijls2009}. The execution of the PNBA*
can be done on a computer with two CPU cores (or two CPUs). One CPU searches forwardly, that is, from the start vertex, it finds a 
path to the end vertex. The other CPU searches backwardly wherein it starts from the end vertex to the start vertex. The PNBA* heuristic function 
need not impose that the used heuristic function is admissible. That is, only consistency is needed. As discussed in Chapter 2, the PNBA* is a 
parallelized version of the NBA*. Since the NBA* employs alternating execution between the two directions, it may be distributed on two different 
CPU cores as demonstrated by Rios and Chaimowicz. The PNBA* can use a data structure such as a priority queue. Each direction will have their own 
priority queue to serve as the open set $\Omega_p$ for $p\in\{1,2\}$, for each process $p$, as stated in the paper by Rios and Chaimowicz.

\begin{definition}
    A vertex $v$ is \textbf{labeled} if the estimate $g(v)$ (the shortest path found so far, that is, an estimate of $\zeta(v_{start}, v)$)
    is finite and $v$ hasn't been expanded.
\end{definition}

The paper establishes the following notational convention to be used for the rest of the paper with regards to the PNBA* algorithm. 
Two processes exist, namely process $p$ in $p\in\{1,2\}$. Whenever $p$ is used, it is implied that $p$ is the process number, i.e the CPU being 
referenced. Let $h_p(v)=\zeta_p(v, v_{goal})$ be the heuristic for estimating the cost from the current vertex $v$ to the goal vertex $v_{goal}$.
Likewise, let $g_p(v)=\zeta_p(v_{start}, v)$ be shortest path so far. The shared set $\mathcal{M}$ contains the vertices in the middle of the two searches.
Initially, all vertices are in $\mathcal{M}$. The shared variable $\mathcal{L}$, the best solution so far, initially set to $\infty$ since 
the algorithm will need to minimize $\mathcal{L}$ as small as possible. The variable $F_p$ is the lowest $f_p$-value of $\Omega_p$ where $f_p(v)=h_p(v)+g_p(v)$.
The variables $f_p$, $g_p$, and $F_p$ can be written only by process $p$ but can be read by both. Note that it is best to implement the open sets $\Omega_p$.

Though an abuse of notation, if $p\in\{1,2\}$ then $\neg:\{1,2\}\to\{1,2\}$ is described as follows:
$$
\neg p=\begin{cases}
    1 & \text{if } p=2 \\ 
    2 & \text{if } p = 1.
\end{cases}
$$

\begin{algorithm}[H]
    \caption{PNBA* (Parallel New Bidirectional A*) Search Algorithm}
    \begin{algorithmic}
        \REQUIRE $G$, the graph 
        \REQUIRE $u$, start vertex and $v$, the end vertex
        \STATE Let $\mathcal{M}$ contain all vertices in $G$ and for all $m\in\mathcal{M}$, let $g_p(m)=\infty$.
        \STATE Let $s_1=t_2=u$ and $s_2=t_1=v$.
        \STATE Set $g_p(s_p) = 0$ and add $s_p$ into $\Omega_p$.
        \STATE From here on, all instructions are run in parallel.
        \WHILE {$\Omega_1\neq\varnothing$ or $\Omega_2\neq\varnothing$}
            \STATE Get the vertex $n\in\Omega_p$ with the minimum $f_p(n)$.
            \IF {$n\in\mathcal{M}$}
                \IF {$f_p(n)<\mathcal{L}$ and $g_p(n)+F_{\neg p}-h_p(n)<\mathcal{L}$}
                    \FORALL {$m$ is a neighbor of $n$ and $m$ is not expanded}
                        \IF {$m\in\mathcal{M}$ and $g_p(m)>g_p(n)+\zeta_p(nm)$}
                            \STATE Set $g_p(m)=g_p(n)+\zeta_p(nm)$
                            \STATE Set $f_p(m)=g_p(m)+h_p(m)$
                            \IF {$m\in\Omega_p$}
                                \STATE Remove $m$ from $\Omega_p$ 
                            \ENDIF
                            \STATE Insert $m$ to $\Omega_p$.
                            \IF {$g_1(m)+g_2(m)<\mathcal{L}$}
                                \STATE Lock $\mathcal{L}$ to ensure that $F_p$ and $L$ are monotonic.
                                \IF {$g_1(m)+g_2(m)<\mathcal{L}$}
                                    \STATE Set $\mathcal{L}=g_1(m)+g_2(m)$.
                                \ENDIF
                                \STATE Unlock $\mathcal{L}$ after the update.
                            \ENDIF
                        \ENDIF
                    \ENDFOR
                \ENDIF
                \STATE Remove vertex $n$ from $\mathcal{M}$.
            \ENDIF
            \IF {$\Omega_p\neq\varnothing$}
                \STATE Set $F_p$ with the lowest $f_p$-value in $\Omega_p$.
            \ENDIF
        \ENDWHILE
    \end{algorithmic}
\end{algorithm}

\subsection{HDA*}
Unlike the PNBA*, the Hash-Distributed A* Search can employ the use of as many CPU cores as desired.\cite{Kishimoto2009} One 
of the main advantages of this is that it can easily be implemented on a GPU with hundreds or thousands of cores, thereby, 
execution would be faster. In the PNBA*, each process has different closed and open sets but in the HDA*, 
these sets are a parallel data structure shared among each CPU cores. However, there is a hash function $k:V(G)\to P$ where 
$V(G)$ is the set of vertices in $G$ and $P$ is the set of processors. When a new vertex is found and the hash function is computed for 
the vertex, the algorithm determines which CPU core should own the vertex based on the hash function.\footnote{Like how a hashmap works}
Thereby, while the open and closed sets are implemented as a parallel data structure, each processor owns a space 
in the open and closed set denoted by $\Omega_p$ and $\Gamma_p$ respectively for processor $p$, based on the hash key.

\begin{algorithm}[H]
    \caption{HDA* (Hash-Distributed A*) Search Algorithm}
    \begin{algorithmic}
        \REQUIRE $G$, the graph 
        \REQUIRE $u$, the start vertex 
        \REQUIRE $v$, the goal vertex
        \STATE Let $\Omega$ be the open set and $\Gamma$ be the closed set. Let $f$ be the evaluating function.
        \STATE Add $u$ to $\Omega$
        \STATE Everything from here on is run in parallel. Each processor $p$ runs the while loop below.
        \WHILE {global $\Omega\neq\varnothing$}
            \IF {$p$ has a new vertex $a$ in its message queue.}
                \IF {$a\notin\Gamma_p$}
                    \STATE Add $a$ to $\Omega_p$.
                \ENDIF
            \ELSE
                \STATE Get the vertex $a$ with the minimum $f(a)$ from $\Omega_p$.
                \STATE Let $N$ be the set of neighbors of $a$ not in $\Gamma_p$.
                \STATE Compute hash key $k(n)$ for all $n\in N$
                \STATE Let $p'$ be the processor that owns the hash key $k(n)$.
                \WHILE {The message queue of $p'$ is locked by another processor}
                    \STATE Wait
                \ENDWHILE
                \STATE Lock $p'$ message queue
                \STATE Send $n$ to $p'$
                \STATE Unlock $'$ message queue.       
            \ENDIF
        \ENDWHILE
    \end{algorithmic}
\end{algorithm}

\section{Haskell Parallel Runtime}
Haskell's parallel execution is implicit and can be done in a monadic environment.\cite{Marlow2013}
Since Haskell is a functional program, the task of parallelization is made easy by the runtime 
by \lstinline{rpar} and \lstinline{rseq} and by specifying strategies on how haskell should 
parallelize the task. However, the Glasgow Haskell Compiler (GHC) runtime may or may not parallelize 
the function at all, this is because \lstinline{rpar} sparks the function for parallelization, that is,
there is a potential that it may be parallelized.\cite{Marlow2005} A common problem for parallelization 
in Haskell is that the runtime garbage collection might take more time compared to the actual algorithm 
execution, therefore decreasing efficiency significantly. This can be mitigated and prevented by analyzing 
the granularity of the data structures. For example,
\begin{lstlisting}[language=Haskell]
    -- We have a list of 10000 elements
    xs = [0..10000]
    -- square all elements of xs
    xs' = (^2) <$> xs
    -- execute in parallel
    runEval (evalList xs')
\end{lstlisting}
This example is inefficient since the GHC runtime will assign each element of \lstinline{xs} to a different 
thread, thereby increasing CPU overhead and garbage collection since the inexpensive task of squaring 
numbers is undermined by the more expensive task of garbage collection. A better implementation would be 
\begin{lstlisting}[language=Haskell]
    -- We have a list of 10000 elements
    xs = [0..10000]
    -- square all elements of xs
    xs' = (^2) <$> xs
    -- execute in parallel
    runEval (parListChunk 5000 xs')
\end{lstlisting}
In this example, the GHC runtime will divide the list into two and have two different processors 
evaluate the two sublists. This way, the processor assignment and garbage collection overhead will be minimized.

Hence, in Haskell, problems of parallelization switched from the implementation details such as deadlocks, 
race conditions, and others commonly encountered in other languages, to that of the dividing the task into 
chunks that GHC will be performant.

\section{Maze Generation}
The mazes were generated using Python and exported to a text file using Depth-First-Search algorithm.
Since the depth-first search algorithm is guaranteed to produce a solvable maze, the researchers chose to use 
depth-first search when generating the maze.

Two versions of the DFS algorithm were used. One of them were randomized node selection while exploring the grid.
The second was to choose randomly whether the selected node will remain open, that is, have no wall, or to be closed.
The randomization of the DFS algorithm will generate a more robust and unpredicted maze generation that ensures 
there are multiple ways to solve a given maze.\cite{CLRS}

% Likewise, we discuss the graph representation we will use (adjacency list vs adjacency matrix)
% and how Haskell's parallelization works with respect to the Haskell Runtime, such as how 
% sparks are created which may or may not be parallelized at all! 

% We're still debating whether to use the Categorical Abstract Machine so there's no loss of generality 
% over different programming languages. If so, we will discuss a little bit of the syntax and notation for formality.
% If not, we will skip this step and proceed with using a concrete implementation of Haskell.

% Here, we discuss the original A* algorithm which is implemented in abstracted Haskell and two different parallel A* algorithm in 
% a Python-like pseudocode, specifically, PBA* and HDA*.
% A superstring is simply a string over
% some alphabet for which given a set of string from the same
% alphabet, the latter's members are all substrings of the former.
% To understand the SCS problem, it is best to assume some important concepts
% related to the problem itself. First, let
% us assume some set $R = \{x_1, \ldots , x_n\}$ as
%  a set of strings (or blocks) over some alphabet $\Sigma$. 
% Formally, we define a $superstring$ of $R$ as a 
% string $w$ containing each $x_i \in R$, as a substring.

% Following are the elementary operations associated 
% with the construction of a superstring.
% The $overlap(u, v)$ of two strings $u$ and $v$ 
% is the maximum overlap between $u$ and $v$. That is to say,
% the longest suffix of $u$ (in terms of characters) that is a prefix of $v$.
% The $prefix(u, v)$ of $u$ and $v$ is the prefix of $u$
% obtained by removing its overlap with $v$.
% Lastly, $merge(u, v)$ is the
%  concatenation of $u$ and $v$ with the overlap appearing
% only once.

% As an example, say we have, $\Sigma = \{a, b, c\}$ and $R = \{cbcaca, cacac\}$.
% The string $cacacbcaca$ is a superstring of $R$ while the string
% $cbcacac$ is the shortest common superstring. The
% following relations also hold: 

% \begin{tabular}{p{0.5cm}l}
% &$overlap(cbcaca, cacac) = caca$, \\
% &$overlap(cacac, cbcaca) = c$ ,\\
% &$prefix(cbcaca, cacac) = cb$,\\
% &$merge(cbcaca, cacac) = cbcacac$.
% \end{tabular}

% A $superstring$ $w$ is also defined as
%  the string $prefix(x_1, x_2) \cdot prefix(x_2, x_3) \cdots prefix(x_n, x_1) \cdot overlap(x_n, x_1)$.
% That is, a $superstring$ is simply
%  a concatenation of all the strings ``minus'' the overlapping duplicates.
%  Apparently, each superstring of a set of strings defines a permutation of the set’s elements.
% Conversely, every permutation of the set’s elements corresponds to a single superstring
			 
% Our interest however lies on defining the SCS problem.
% Essentially, the SCS problem is: Given a set of strings $R$
% and a positive integer $K$, does $R$ have a superstring of length $K$?

% Figure \ref{fig:scsdp} shows the SCS Decision Problem following the template
% provided by Garey and Johnson \cite{GareyJohnson1979}.

% \begin{figure}[ht!]
% \centering
% \fbox{
% \begin{tabular}{p{7.5cm}}

% INSTANCE: \\
% Finite alphabet $\Sigma$, finite set $R$ of strings from $\Sigma^*$ and a positive integer $K$. \\
% QUESTION: \\
% Is there a string $w \in \Sigma^*$ with $\vert w\vert \leq K$ such that each string $x\in R$ is a substring of w, 
% i.e. $w=w_0 x w_1$ where  $w_0,w_1\in\Sigma^*$? \\
% \end{tabular}
% }
% \caption{SCS Decision Problem }
% \label{fig:scsdp}
% \end{figure}


% In the Compendium of NP Optimization problems published by \cite{AusielloEtAl2000} and in the list of 
% NP-Complete Problems published by \cite{GareyJohnson1979} the SCS problem appears under under Storage and Retrieval (SR).
%%%%%%%%%%%%%%%%%%%%%%%%%%%%%%%%%%%%%%%%%%%%%%%%%%%%%%%%%%%%%%%%%%%%%%%%%%%
\chapter{Methodology}
%%%%%%%%%%%%%%%%%%%%%%%%%%%%%%%%%%%%%%%%%%%%%%%%%%%%%%%%%%%%%%%%%%%%%%%%%%%

\section{Experimentation}
The researchers would write multiple programs in Haskell to solve combinatorial problems.
Likewise, previous papers from HDA* and PNBA* will be reimplemented using Rust due to its 
Algebraic Data Types, much like Haskell, and its zero-cost abstraction features, which will
be utilized as to compare the performance of the algorithm if implemented in either 
programming disciplines.\cite{Kishimoto2009,Rios2011}

\section{Data Gathering and Documentation}
Run times on different systems with different CPU cores will be recorded, such as 
idle times, thread sparking, and garbage collection, using Threadscope.\cite{ThreadScope} Results will be recorded 
in a tabular format and plotted with its performance metrics with respect to the problem size 
and performance for each number of CPU cores.
Comparisons between both CPU run times and CPU core activities will be made to examine each
problem for each alorithm for both the Haskell implementation and Rust implementation. 
Metrics will be plotted in the same figure for ease of readability.

\section{Combinatorial Problems}

To make the results of the paper more reliable, multiple combinatorial problems will be tested
upon and recorded. The problems are handpicked from previous researches and some of the problems 
are classic combinatorial problems that might provide interesting results when testing the program.
\begin{itemize}
    \item \textbf{Freecell} is a solitaire-like card game employing the standard 52-deck card. 
        An optimal solution is when the program finds a way to stack all the cards in their 
        corresponding stack with the same suit and in a chronological order.
    \item \textbf{Sokoban} is a puzzle video game in which the player's requirement to win is to 
        push the boxes into corresponding storage locations. Since Sokoban has many levels that 
        rage from easy to difficult, these levels of difficulty can be used as a \emph{problem size} 
        when recording the experiment. An optimal solution guarantees the least number of moves 
        to push the crates and cover their corresponding storage locations.
    \item \textbf{n-queens problem} is a classic combinatorial problem where there is a standard 
        chess board and $n$ queens positioned in a way such that no two queens are threatening each other.
        The $n$-queens problem is a generalization of the eight queens problem which has a total of 
        92 solutions. 
    \item \textbf{Sudoku} is another classical combinatorial game with some cells containing numbers 
        and the goal is to solve the remaining cells.    
    \item \textbf{Knapsack Problem} is one of the classic optimization problems where the reward is maximized 
        while the cost is minimized. The problem stated here will be the 0/1 Knapsack Problem which is itself 
        solvable by a branch and bounding algorithm like the A*.    
\end{itemize}


%%%%%%%%%%%%%%%%%%%%%%%%%%%%%%%%%%%%%%%%%%%%%%%%%%%%%%%%%%%%%%%%%%%%%%%%%%%
\chapter{Results}
%%%%%%%%%%%%%%%%%%%%%%%%%%%%%%%%%%%%%%%%%%%%%%%%%%%%%%%%%%%%%%%%%%%%%%%%%%%

The researcher conducted an experiment to validate the results
in \cite{Zaritsky2004}'s work. Following is the description of the 
experiment done.

The input strings used in the experiment were generated in a
 manner similar to the one used in DNA sequencing. That is, a random
string is generated, duplicated a predetermined number of times,
 and the copies are randomly divided into blocks of a given size. 
 The set of all these blocks is the input to the SCS problem. 
 The reasons for choosing this type of input has been extensively
 explained in \cite{Zaritsky2004}.
Since our random string is a sequence of characters \texttt{A},\texttt{C},\texttt{G}, and \texttt{T}, we must 
transform this into an equivalent binary string. To do this, we simply 
assign each letter a unique two-bit representation, that is, 
\texttt{A = 00}, \texttt{C = 01}, \texttt{G = 11}, and \texttt{T = 10}. 

The parameters used in the input generation is as follows:

\begin{tabular}{p{0.5cm}l}
&\textit{Size of random string}: 250 bases or characters\\
&\textit{Minimal block size}: 20 characters\\
&\textit{Maximal block size}: 30 characters\\
&\textit{Number of duplicates created from the random string}: 5\\
\end{tabular}

Let $l$ denote the length of the derived string in the
genetic algorithm case. Let $m$
denote the number of blocks not covered by the derived string, 
and let $b$ denote the maximal block size
(30, in our case). The fitness value, $f$ , of an individual
is computed as follows:
\[f = \dfrac{1}{l + m*b}\] 
This fitness function above drives evolution towards
shorter superstrings covering as many blocks as possible.

We performed 50 randomly generated problem instances.
On each problem instance each type of genetic algorithm was executed
twice and the better run of the two was used for statistical purpose.
The results are summarized in Table \ref{tbl:resultsga} and is presented graphically in Figure \ref{fig-results}.


\begin{table}
\centering
\begin{tabular}{|c|c|c|}
\hline
Number  & Average Superstring  & Average  \\
 of& Length (Rounded to & Runtime \\
Generations& Nearest Units) & (in Seconds) \\
\hline
\hline
100		& 760	& 34.57		\\
250		& 596	& 74.20		\\
500		& 526	& 139.70	\\
1000	& 482	& 267.09	\\
5000	& 421	& 1038.73	\\
10000	& 413	& 1943.55	\\
20000	& 409	& 3700.57	\\
\hline
\end{tabular}
\caption{
Average superstring length and runtime of our Genetic Algorithm
on 50 randomly generated problem instances. Each input set contains
50 blocks.}
\label{tbl:resultsga}
\end{table}


\begin{figure*}[ht!]
\centering
\fbox{
\scalebox{0.55}{
\includegraphics{result-gavsgreedy.pdf}
}}
\caption{Best superstring as a function of generations. Each point in the figure represents the average of 50
runs on 50 different randomly generated problem instances. For each such instance, two runs were performed, the better
of which was considered for statistical purposes}
\label{fig-results}
\end{figure*}

Note that our results seem to disagree with that of \cite{Zaritsky2004}.
In contrast with the latter's findings, our tests shows that
the greedy algorithm outperformed the genetic algorithm both in terms of finding superstrings
with minimal length and the the time it takes to do so.

On the same set of test cases, the greedy algorithm
produced superstrings having an average of 361 characters, 
in an average of runtime 130.64 milliseconds. This is far better than
the 409 average superstring length that the genetic algorithm
derived in 20000 generations in more or less an hour of operation. 
Note that the average length of the superstring derived by the greedy algorithm is
less than the conjectured factor-2 performance guarantee.
Recall that the length of the initial (reference) string is 250.

Still it is good to note that the genetic algorithm approach
in solving the SCS problem converges. That is, we see a better solution
yield as time or generations progress. 
%%%%%%%%%%%%%%%%%%%%%%%%%%%%%%%%%%%%%%%%%%%%%%%%%%%%%%%%%%%%%%%%%%%%%%%%%%%
\chapter{Contributions and Recommendations}
%%%%%%%%%%%%%%%%%%%%%%%%%%%%%%%%%%%%%%%%%%%%%%%%%%%%%%%%%%%%%%%%%%%%%%%%%%%

\section{Summary of Contributions}


\section{Recommendations}


There could be several reasons why our tests did not yield the ``same'' results as that of \cite{Zaritsky2004}'s.
The author claims that this may be due to the myriad of parameters used in evolving
individuals in each step of the genetic algorithm and the large number of possible candidates or values that we can set these parameters with. 
For example, we may claim that the appropriate choice of the recombination technique is necessary. In \cite{Zaritsky2004},
a two-point crossover was employed. Thinking that this may be less of a factor, the author proceeded with just
one-point cross in implementing the genetic algorithm for the experiment.
Another could be in choice of the probabilities associated with recombination, elitism, and mutation rates.
The author did only consider what was stated in the base paper without performing tests to validate
whether this values are appropriate and could still complement the change made in some of the assumptions.
As such, more empirical tests could be done to appropriately estimate these parameters. 

If indeed DNA sequence assembly is the task at hand, lots of other techniques are now available.
One may consider for example representing the reads or DNA fragments in a de Bruijn graph
as opposed to the overlap graph presented in this paper. 
Assembling reads  in a de Bruijn graph 
reduces the problem to a fragment assembly problem that can be 
formulated as the goal to find a trail or Eularian path that visits 
each edge (read or fragment) in the (de Bruijn) graph exactly once. 
Such, somehow makes the assembly process much ``easier'' since Eularian path
construction is known to be solvable in deterministic polynomial time.



\appendix
% \chapter{Evaluation Tool}

Evaluation tool used goes here.

% \chapter{Sample Reports}

Describe and discuss the details of sample reports here.

% \chapter{Sample Input/Output}

Describe and discuss the details of sample I/O here.

\chapter{Code Listing}

\lstinputlisting{src/scs-greedy.cpp}

% \include{appendices/usersguide}

\nocite{*}
\bibliographystyle{siam}
{
\singlespace
\bibliography{references}
}

\begin{vita}
JR is BS Information Technology student of the
Department of Computer Science at the Ateneo de Naga University.
\end{vita}



\end{document}
